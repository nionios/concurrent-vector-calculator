% Only difference in this pagestyle is the number is from the section
\newpagestyle{conclusions}{
    \setheadrule{.4pt}% Header rule
    \sethead{\small{Section \thesection}}% left
    {\emph{\small{Dionisis Nikolopoulos - Open Source Search Engines}}}%    center
    {}%    right
    \setfoot {}
    {\thepage}
    {}
}
\pagestyle{conclusions}
\section{Conclusions}
Seeing how search engines are now integrated into our daily lives through the
applications that we use, every facet of our personality and interests can be
deducted through the information that we give away for free.

In recent years, the internet has unified the voices of people that object to
the practices used to gather information and profit from their sale.
This dissent from the contemporary mainstream search engines is now a staple of
the larger privacy-conscious community online.
This community has a large amount of crossover with the open source community,
as is evident by tools like Searx which is fully open source; and even
DuckDuckGo, which is increasingly open source.

It seems that open source technologies and privacy-first applications and tools
very often coincide; and with good reason.
Open source allows for the literal maximum of access to the internal
functions of the program, an ability that cannot be underestimated, given the
demand for privacy and the brewing amount of distrust for big tech companies.

Overall, open source search engines are spawning a new market.
Today, they might be obscure compared to mainstream competitors, but in a few
years, with refinements and the with public interest continuing to develop
steadily, open source engines could be a part of a new era in internet privacy.