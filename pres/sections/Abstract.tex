% NOTE: In this file all the ToC entries are manually added because they
% ```   dont show up on the ToCs with the * method otherwise
\thispagestyle{empty}
\begin{center}
    \huge{
        \textbf{
            \color{brown}Concurrent Vector Calculator\color{black}
            \vspace{5mm}
        }
    }\\
    \large{Ένα πρόγραμμα υπολογισμού διανυσμάτων με χρήση RPC.}
    \vspace{5mm}\\
  %  \includesvg[scale=5]{../svg/cover}\vspace{5mm}\\
    \small{\emph{
    Διονύσης Νικολόπουλος, ice18390126@uniwa.gr
    \\
    Τμήμα Μηχανικών Πληροφορικής και Υπολογιστών
    \\
    Πανεπιστήμιο Δυτικής Αττικής, Ελλάδα, Αθήνα
    \\
    Απρίλιος - Μάιος 2022
    \\
    }}
    \vspace*{5mm}
\end{center}
\footnotesize
Keywords: \emph{Γλώσσα προγραμματισμού C, Remote Procedure Calls, Κατανεμημένα
Συστήματα}
\normalsize


\section*{Θέμα και Λεπτομέρειες Υλοποίησης}
\begin{addmargin}[2em]{3em}% 2em left, 3em right
\footnotesize{
    Ζητούμενο της εργασίας ήταν η κατασκευή ενός προγράμματος τέλεσης 3 πιθανών
    διαδικασιών (Εύρεση Μέσου όρου, Εύρεση μικρότερου και μεγαλύτερου στοιχείου,
    Γινόμενο διανύσματος με έναν αριθμό).
    \\
    Όλα τα ερωτήματα της εκφώνησης, όπως αυτή φαίνεται στον φάκελο exc/ του
    repository της εργασίας, έχουν πραγματοποιηθεί.
    \\
    Το πρόγραμμα δημιουργήθηκε και δοκιμάστηκε σε περιβάλλον Arch Linux
    με kernel 5.17.5-arch1-1, με gcc version 11.2.0, rpcgen (rpcsvc-proto) 1.4.3
    \\
    Ο κώδικας που γράφθηκε υπόκειται στην άδεια GNU GPLv3.
    \\
    Η Εργασία αυτή πραγματοποιήθηκε στο πλαίσιο του μαθήματος των Κατανεμημένων
    Συστημάτων του προγράμματος σπουδών Πανεπιστημίου Δυτικής Αττικής, τμήματος
    Μηχανικών και Πληροφορικής Υπολογιστών.
    \\
    \color{blue}
    Κάθε αρχείο αυτής της εργασίας, συμπεριλαμβανομένου και του παρόντος
    εγχειριδίου βρίσκονται αποθηκευμένα σε ένα αποθετήριο στο GitLab,
    \href{https://gitlab.com/nnis/concurrent-vector-calculator}
    {στον υπερσύνδεσμο αυτό.}
    \color{black}
    \\
    Εκεί φαίνεται αναλυτικά το πως και για πόσο έγινε η δουλειά πάνω
    στο πρόγραμμα.
}
\end{addmargin}
